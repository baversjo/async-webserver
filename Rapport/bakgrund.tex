\section{Bakgrund}

När vi först skulle besluta oss för vad vi ville göra i detta projekt användes hemsidan som grund, som sen tidigare presenterat gamla projekt. Något som inte fanns med och som till stor del kräver att man implementer våra kunskaper i nätverksprogrammering är någon typ av server av mera komplex typ. Vi tog ett gemensamt beslut om att det vi vill skapa är en HTTP server som ska ligga till grund för fortsatta implementationer av plugins som ska extenda våran HTTP server.
Detta räckte inte för att uppfylla kraven som projektet skulle täcka eftersom en HTTP server kan göras mer eller mindre simpel.
Det vi istället gjorde är att gå från trådbaserad synkron server till att implementera en asynkron server.
Det som gör en asynkron server bättre är att den kan hantera flera HTTP request samtidigt och som resultat av detta ska flera filer kunna läsas samtidigt. Detta tillhandager även enkelhet för implementeraren av applikationslogik då han slipper tänka på trådar, vilka i en traditionell server tar upp en request per tråd. Detta gör det väldigt kostsamt och ineffektivt för servern. Det som händer när en request skickas till en asynk webbserver är att requesten behandlas som en instruktion som insturerar kerneln att utföra en viss handling. Detta är en så kallad zero copy - transfer to funktion som gör servern effektic. När man då ska hämta en fil från servern kommer requsten att instruera kerneln att hämta filen från hårddisken som sedan sparas i serverns RAM minne för direkt åtkomst. Under denna process kommer servern att använda sig av blocking I/O vilket behövs för att få tillgång till filen. När man använder sig av blocking I/O så blockeras all kommunikation tills dess att filen skickats tillbaka.
Detta gör en server långsam men detta är tvunget då man på något sätt måste nå filen första gången. När filen sedan sparats i RAM minnet kan resterande anslutningar använda sig av non blocking I/O vilket tillåter flera request (målet med vår server) samtidigt. Med en non blocking I/O blockeras inget vilket betyder att flera anslutningar kan fråga servern efter olika saker. Det som som händer då är att kerneln skickar tillbaka en callback som anropar rätt anslutning så att den kan få det efterfrågade.
För att implementera detta använde vi oss av java 7 asynkrona fil-API samt java.nio. Java.nio är en samling av APIs utvecklat för att hantera centraliserade I/O operationer. 

