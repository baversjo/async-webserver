\section{Modell}
Vi använder huvudsakligen en klass Server för att hantera anslutningar. I Server initieras i sin tur ett Clientobjekt för varje anslutning. Clientobjekten delegeras till en tråd och lagras sedan i en HashMap lokalt i respektive tråd.  För att öka effektiviteten finns det lika många trådar som det finns kärnor. Requesten från en klient parsas i Client och ett response instansieras, exekveras genom alla FileMiddleWare. FileMiddleWare lägger i sin tur till allt nödvändigt i Responseobjektets headers, och slutligen skickar den även ut filen som efterfrågats. \\
Den mest använda datastrukturen är HashMap som används till att lagra både klienter och headers. För att trådarna ska vara så effektiva som möjligt är även klienterna sorterade i en priorityqueue i vardera tråd.
Det som händer när en request skickas till vår webbserver är att requesten behandlas som en instruktion som insturerar kerneln att utföra en viss handling. Detta är en så kallad zero copy - transferTo funktion som gör servern effektiv. När man då ska hämta en fil från servern kommer requesten att instruera kerneln att hämta filen från hårddisken som sedan sparas i serverns RAM-minne för direkt åtkomst. Under denna process kommer servern att använda sig av blocking I/O vilket behövs för att få tillgång till filen. När man använder sig av blocking I/O så blockeras all kommunikation tills dess att filen skickats tillbaka.
Detta gör en server långsam men detta är tvunget då man på något sätt måste nå filen första gången. När filen sedan sparats i RAM-minnet kan resterande anslutningar använda sig av non blocking I/O vilket tillåter flera request (målet med vår server) samtidigt. Med en non blocking I/O blockeras inget vilket betyder att flera anslutningar kan fråga servern efter olika saker. Det som som händer då är att kerneln skickar tillbaka en callback som anropar rätt anslutning så att den kan få det efterfrågade.\\
För att implementera detta använde vi oss av java 7 och java.nio. Java.nio är en samling av APIs utvecklat för att hantera centraliserade I/O operationer. 

